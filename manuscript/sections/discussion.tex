% SPDX-FileCopyrightText: 2023 Iegor Riepin, Tom Brown
%
% SPDX-License-Identifier: CC-BY-4.0

\textbf{Our results in perspective --} This work contributes to and builds on two streams of literature: one is concerned with the flexibility and environmental impact of datacenters, the another with the means, costs and implications of 24/7 carbon-free energy matching.

Regarding the first literature stream, our findings are in line with those of existing model-based studies on the flexibility potentials offered by geographically distributed datacenters: load-shifting flexibility facilitates the integration of renewable energy sources and reduces the carbon footprint of datacenters \cite{zhengMitigatingCurtailmentCarbon2020, mahmudDistributedFrameworkCarbon2016, wangGreenawareVirtualMachine2015, kimDataCentersDispatchable2017, liuGeographicalLoadBalancing2011, kellyBalancingPowerSystems2016, lindbergEnvironmentalPotentialHyperScale2021}.
Although opportunities and challenges of datacenter flexibility have drawn considerable attention in the literature, space-time load-shifting flexibility has not been studied in the context of 24/7 CFE matching pursued by companies aiming to run their facilities at 0~gCO$_2$/kWh.
Since hourly matching of consumption with carbon-free electricity requires and implies renewable energy procurement through \gls{ppa}s and thus less electricity procurement from the grid, we argue that grid signals such as average carbon emission intensity or locational electricity prices have low value for informed use of load-shifting flexibility in the context of the 24/7 CFE matching.
In order to address this challenge, we identify three signals that can enable companies to shift load across space and time effectively.
Moreover, we demonstrate how companies can achieve significant gains in energy efficiency and affordability of 24/7 CFE matching by including these signals into their energy procurement and load-shaping strategies.

Regarding the second literature stream, our results indicate that space-time load-shifting increases access to clean electricity and provides flexibility to consumers for matching demand with carbon-free electricity.
As a result, datacenters and other commercial and industrial consumers who have flexible demands can achieve high degrees of carbon-free energy matching with lower resources and thus be more cost-effective.
The system decarbonization impact of 24/7 CFE procurement indicated by the model-based studies \cite{xu-247CFE-report,riepin-zenodo-systemlevel247} could therefore be amplified with greater participation.

Finally, our results follow the general consensus on the benefits of geographical load balancing in highly renewable electricity networks \cite{schlachtbergerBenefitsCooperationHighly2017}.
However, the benefits of connecting dispersed regions at continental scales are always weighed against the high costs and long lead times associated with grid expansion and reinforcement.
It is unique to the datacenter industry that geographical load-shifting flexibility can be achieved at low costs and over long distances by harnessing local differences in the quality of renewable resources and renewable generation profiles.

\textbf{A broader context --} The first commitments to 24/7 CFE matching initiated by the datacenter industry can help companies from other sectors by establishing procurement methods and standard practices for 24/7 CFE matching. \cite{xu-247CFE-report}.
In the same way, early efforts to consider potentials of available load flexibility within energy procurement and load-shifting strategies can help a broad range of companies that are seeking to lower their carbon footprints.
In the event that datacenter companies, as well as other commercial and industrial consumers with flexible loads can achieve high degrees of 24/7 CFE matching at moderate costs, more actors may be willing to commit to it.
Therefore, the system decarbonization impacts associated with the 24/7 CFE procurement \cite{riepinMeansCostsSystemlevel2023} could be amplified with greater participation while requiring fewer resources.

The regulatory environment can play a crucial role in unlocking the full potential of sustainability-driven innovations.
For example, the European Union's recent Corporate Sustainability Reporting Directive \cite{DirectiveEU20222022} aims to enhance the transparency of sustainability reporting by mandating a broad range of companies to disclose key energy and sustainability metrics.
Such regulation foster greater industry-wide transparency and enhance the understanding of corporate and industrial carbon footprints.

Additionally, even though it is not the focus of the present study, our results provide necessary methodological foundation for studying strategic locations for new datacenters by companies interested to reduce or eliminate completely their carbon footprint. It is therefore important to consider the factors facilitating resource-efficient and cost-effective 24/7 CFE matching identified by this study when choosing optimal locations for datacenters.

\textbf{Critical appraisal and further work --} The study design includes simplifications and assumptions inherent to energy optimization models. The results should thus be seen with caution, i.e., as model-derived insights rather than quantitative projections.

One of the central assumptions of the model experiment is that datacenters achieve a certain degree of flexibility in their workloads. While we control the share of flexible workloads exogenously (see \nameref{sec:methods}), the actual flexibility of datacenters is not known with certainty and can vary significantly among facilities.
As specific data becomes available, the model can be calibrated to reflect the actual flexibility potential.

Additional empirical research is required to quantify the costs and benefits of utilizing demand flexibility in the \gls{ict} industry.
Further studies could usefully explore the costs and technical potentials of achieving a certain share of flexible workloads, which are not considered in this study.
By including implicit flexibility costs, the benefits of flexibility can be quantified more accurately.
An empirical improvement could also address technical aspects and properties of flexible workloads, such as power usage ramping up and down, reliability, and performance constraints.

In the context of 24/7 CFE matching, further research could explore the role of spatio-temporal load-shifting flexibility during extreme weather events, such as solar and wind droughts, and examine its implications for 24/7 CFE matching costs. Additionally, investigating the hedging value of spatio-temporal load-shifting flexibility in combination with advanced energy technologies, such as long-duration energy storage, would be valuable. This line of research is particularly relevant, as 24/7 CFE procurement can create an early market and accelerate the deployment of advanced technologies, as shown in recent studies \cite{xu-247CFE-report,riepinMeansCostsSystemlevel2023}. Further studies might focus on a complex interaction of geographical load shifting with the long-duration storage \cite{riepinValueSpacetimeLoadshifting2023}.
