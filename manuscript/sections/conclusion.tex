% SPDX-FileCopyrightText: 2023 Iegor Riepin, Tom Brown
%
% SPDX-License-Identifier: CC-BY-4.0

This work explores the role of space-time load-shifting flexibility provided by geographically distributed datacenters in achieving 24/7 carbon-free energy (CFE) matching and identifies signals relevant for effective load shifting.

We develop an optimization model to simulate a network of datacenters managed collectively by a company pursuing 24/7 carbon-free energy matching objective.
The model provides users with the ability to specify datacenter locations within the European electricity system, along with the flexibility potentials, among other parameters.
Through energy system modeling, we isolate three individual signals that are crucial for effective space-time load-shifting: (a) varying quality of renewable energy resources across datacenter locations, (b) low correlation between wind power generation over long distances due to different weather conditions, and (c) lags in solar radiation peak due to Earth rotation. Each of these signals contributes to effective load shifting, although how each signal is weighed varies with datacenter locations and time of year.
Further, we demonstrate that energy procurement and optimal load-shifting decisions made based on these signals facilitate the cost-efficiency of clean computing. With every percentage of flexible load added, 24/7 CFE matching costs fall by 1.29$\pm$0.07~\euro/MWh.

Overall, our results indicate that space-time load-shifting flexibility increases access to clean electricity and gives consumers more options to match demand with carbon-free electricity.
By utilizing demand flexibility, truly clean computing can be made more affordable and resource-efficient.
Finally, early efforts by the datacenter industry can help and pave the way for a broad range of companies by establishing standard methods and practices for load-shifting flexibility in the context of 24/7 CFE matching.