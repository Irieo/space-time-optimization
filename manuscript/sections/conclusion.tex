% SPDX-FileCopyrightText: 2023 Iegor Riepin, Tom Brown
%
% SPDX-License-Identifier: CC-BY-4.0

This work explores how spatio-temporal load-shifting flexibility among geographically distributed datacenters can facilitate 24/7 carbon-free energy (CFE) matching. Through energy system modeling, we identify three key signals for effective load shifting: (i) variations in the quality of renewable resources, (ii) low correlation in wind generation over long distances, and (iii) time lags in solar radiation peaks due to Earth's rotation.

Our findings demonstrate that leveraging load flexibility maximizes the utility of high-quality renewable resources, enables cost-effective use of wind generation across uncorrelated regions, and optimizes solar energy utilization by exploiting time zone differences. Generalizing these results across multiple locations within Europe, we observe that spatio-temporal load flexibility consistently improves resource efficiency and reduces costs, irrespective of potential locations of datacenters. Every 1\% increase in flexible loads on average decreases costs of 24/7 CFE matching by 1.29$\pm$0.07 \euro/MWh. The optimal load-shaping strategy depends on specific datacenter locations and time of year. Our results also illustrate the diminishing returns of additional load flexibility, which is important for companies seeking to co-optimise their long-term energy procurement and short-term load-shifting strategies.

This work provides practical guidelines for datacenter companies to effectively leverage spatio-temporal load flexibility for truly clean computing. Furthermore, our findings and the open-source optimization model provide valuable insights and practical tools for a wide range of industries striving to achieve carbon neutrality.