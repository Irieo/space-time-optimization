% SPDX-FileCopyrightText: 2023 Iegor Riepin, Tom Brown
%
% SPDX-License-Identifier: CC-BY-4.0

This work explores the role of space-time load-shifting flexibility provided by geographically distributed datacenters in achieving 24/7 carbon-free energy (CFE) matching and identifies signals relevant for effective load shifting.

We develop an optimization model to simulate a network of datacenters managed collectively by a company pursuing 24/7 carbon-free energy matching objective.
The model provides users with the ability to specify datacenter locations within the European electricity system, along with the flexibility potentials, among other parameters.
Through energy system modeling, we isolate three individual signals that are crucial for effective space-time load-shifting: (i) varying average quality of renewable energy resources across datacenter locations, (ii) low correlation between wind power generation over long distances due to different weather conditions, and (iii) lags in solar radiation peak due to Earth's rotation.

In relation to the first signal, we demonstrate that leveraging spatio-temporal load flexibility maximizes the utility of high-quality renewable energy resources. Therefore, the optimal load-shaping strategy entails making informed decisions about where and when to shift loads, based on the average capacity factors of procured renewable energy resources at different locations. Our results indicate that energy costs are reduced at all locations, particularly in those where hourly matching is most challenging. For instance, for a data center located in Germany with partners in Denmark and Portugal, the costs of 24/7 CFE matching decrease from 215~\euro/MWh with no load flexibility to 195~\euro/MWh with 10\% flexible loads, and further to 137~\euro/MWh with 40\% flexible loads.

As for the second signal, we show that spatio-temporal load shifting allows datacenter companies to take advantage of the low correlation between wind power generation over long distances, i.e., to perform \enquote{load arbitrage} between locations with different weather conditions.
The results show that the cost savings are most noticeable for steps of 300-400 km, which is in line with the falloff in wind generation correlation.
Considering a pair of datacenters located approximately 380~km apart in regions with similar wind and solar conditions, the introduction of 10\% flexible loads yields a 9.5\% reduction in energy costs, whereas 40\% flexible loads facilitate a 26.5\% cost savings.
For distances greater than 400~km, wind feed-in correlations are already low, so further increasing the distance between datacenters does not lead to additional cost savings.

Regarding the third signal, we show how spatio-temporal load shifting can be used to exploit the lags in solar radiation peak due to Earth's rotation. Thus, companies seeking 24/7 CFE matching can benefit from the time zone differences between datacenters and the hourly capacity factors of solar power generators at different locations. Solar power can be used more efficiently this way, thereby reducing costs associated with carbon-free energy matching.

Finally, we generalize our results to a broader set of datacenter locations across Europe and show the relationship between the load flexibility and the costs of 24/7 CFE matching. The results reveal that the resource-efficiency and cost-effectiveness of hourly matching are improved by the effective use of space-time load flexibility in all cases. However, the location of datacenters and the time of year affect which of the three signals are most relevant for an effective load-shaping strategy. Further, we show that with every percentage of flexible load added, the costs of 24/7 CFE matching fall on average by 1.29$\pm$0.07 \euro/MWh, which can be interpreted as the marginal value of load flexibility. Our results also illustrate the diminishing returns of additional load flexibility, which is important for companies seeking to co-optimise their long-term energy procurement and short-term load-shifting strategies.

Overall, our results indicate that space-time load-shifting flexibility increases access to clean electricity and gives consumers more options to match demand with carbon-free electricity. Demand flexibility can enhance resource efficiency and reduce costs of truly clean computing. This can amplify the decarbonization effects associated with the voluntary clean energy commitments by encouraging more datacenter companies and other electricity consumers to join the 24/7 CFE movement.