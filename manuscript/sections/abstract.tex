% SPDX-FileCopyrightText: 2023 Iegor Riepin, Tom Brown
%
% SPDX-License-Identifier: CC-BY-4.0

% The 5 things a good abstract needs:
% 1. Introduce the topic: what we know
% 2. State the unknown: what we don't know
% 3. Outline the methods: how we answer the questions
% 4. Preview the findings: what we learned
% 5. Tell us what your work teaches us: what we conclude and why this is important
Increasing energy demand for cloud computing raises concerns about its carbon footprint. Companies within the datacenter sector procure significant amounts of renewable energy to reduce their environmental impact. There is increasing interest in achieving 24/7 Carbon-Free Energy (CFE) matching in electricity usage, aiming to eliminate all carbon footprints associated with electricity consumption. However, the variability of renewable energy resources poses significant challenges for achieving this goal. In this work, we explore the role of spatio-temporal load-shifting flexibility provided by hyperscale datacenters in achieving a net zero carbon footprint. We develop a computer model to simulate a network of geographically distributed datacenters managed by a company pursuing 24/7 carbon-free energy matching. Through energy system modeling, we identify and isolate signals that companies can incorporate into load-shaping strategies to facilitate informed and effective load shifting. We demonstrate how individual signals differ in weight depending on the location of the datacenter and the time of year. We show that optimal energy procurement and load-shifting decisions based on these signals can enhance the resource-efficiency and cost-effectiveness of clean computing. The costs of 24/7 CFE matching are reduced by 1.29$\pm$0.07~\euro/MWh for every additional percentage of flexible load.