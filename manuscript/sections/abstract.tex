% SPDX-FileCopyrightText: 2023 Iegor Riepin, Tom Brown
%
% SPDX-License-Identifier: CC-BY-4.0

% The 5 things a good abstract needs:
% 1. Introduce the topic: what we know
% 2. State the unknown: what we don't know
% 3. Outline the methods: how we answer the questions
% 4. Preview the findings: what we learned
% 5. Tell us what your work teaches us: what we conclude and why this is important

Companies operating datacenters are increasingly committed to procuring renewable energy to reduce their carbon footprint, with a growing emphasis on achieving 24/7 Carbon-Free Energy (CFE) matching---eliminating carbon emissions from electricity use on an hourly basis.
However, variability in renewable energy resources poses significant challenges to achieving this goal.
This study investigates how shifting computing workloads and associated power loads across time and location supports 24/7 CFE matching.
We develop an optimization model to simulate a network of geographically distributed datacenters managed by a company leveraging spatio-temporal load flexibility to achieve 24/7 CFE matching.
We isolate three signals relevant for informed use of load flexibility: (1) varying average quality of renewable energy resources, (2) low correlation between wind power generation over long distances due to different weather conditions, and (3) lags in solar radiation peak due to Earth's rotation.
Our analysis reveals that datacenter location and time of year influence which signal drives an effective load-shaping strategy.
By leveraging these signals for coordinated energy procurement and load-shifting decisions, clean computing becomes both more resource-efficient and cost-effective---the costs of 24/7 CFE are reduced by 1.29$\pm$0.07~\euro/MWh for every additional percentage of flexible load.
This study provides practical guidelines for datacenter companies to harness spatio-temporal load flexibility for clean computing.
Our results and the open-source optimization model offer insights applicable to a broader range of industries aiming to eliminate their carbon footprints.