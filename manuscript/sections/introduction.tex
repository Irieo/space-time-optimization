% SPDX-FileCopyrightText: 2023 Iegor Riepin, Tom Brown
%
% SPDX-License-Identifier: CC-BY-4.0

% Introduction Structure

% 1.1 Opening: ICT sector energy consumption and emissions

Modern technology has impacted almost every aspect of everyday life, leading to a rapidly growing demand for digital services. Internet users have doubled worldwide over the past decade, while traffic has grown 25-fold. \cite{ieaDataCentresData2023}
The significant energy consumption of the Information and Communications Technology (\gls{ict}) infrastructure is becoming a concern from the ecological standpoint.
Global datacenter energy use is estimated to be between 1\% and 2\% of the final electricity demand worldwide in 2020 \cite{davidmyttonHowMuchEnergy2020, masanetRecalibratingGlobalData2020} and is likely to increase rapidly in the future \cite{andraeGlobalElectricityUsage2015}. 
The Greenhouse Gas (\gls{ghg}) emissions footprint of the datacenters and data transmission networks was estimated at 330~million tones (Mt) \co-eq in 2020, equivalent to 0.9\% of global energy-related emissions. \cite{ieaDataCentresData2023, malmodinICTSectorElectricity2023} 
For perspective, the national CO$_2$ emissions of the United Kingdom were around 331~Mt in 2021. \cite{UKnationalstats} 

% **datacenter special features as electricity consumers*

The growing demand for computing drives economies of scale. Large companies, such as Amazon, Google, and Misrosoft are centralizing computing infrastructure into the \enquote{hyperscale} datacenters. 
Such large and highly efficient facilities consist of hundreds of server nodes, that are managed collectively and geographically dispersed throughout the world \cite{ThereAre500}. 

In their role as electricity consumers, hyperscale datacenters have unique characteristics when it comes to demand-side management. 
Some computing jobs and associated power loads are \enquote{flexible}, i.e., the jobs are not time-critical and can be scheduled flexibly without impacting the overall quality of service.
Therefore, datacenter operators have the ability to shift a considerable fraction of computing workloads \textit{in~time} and \textit{in space}. \cite{radovanovicIEEE2023} 
The temporal load shifting implies re-scheduling of flexible workloads to another time point, i.e., delaying computing job execution. 
The spatial load shifting implies migration of flexible compute jobs and associated power loads across physical datacenter locations. 
These characteristics equip datacenter operators with the unique ability to provide space-time load-shifting flexibility to the power grid \cite{zhangRemuneratingSpaceTime2022} and reduce the carbon footprint of their electricity usage. \cite{radovanovicIEEE2023, douCarbonAwareElectricityCost2017}


% Literature review
\textbf{Existing literature --} In response to the significant energy consumption of \gls{ict} infrastructure, its anticipated growth, and the associated economic and environmental concerns, a body of literature has explored how datacenter industry can take advantage of spatial and temporal load-shifting abilities.

The work by Wierman et al.  \cite{wiermanOpportunitiesChallengesData2014} surveyed the forms, opportunities and challenges for data center demand response options, including aspects of temporal load shifting such as workload delaying or shedding and the geographical load balancing. Several papers discussed methods and policies needed to incentivise parcicipation of datacenters in the power grid's demand responce \cite{liuPricingDataCenter2014, zhouTruthfulEfficientIncentive2020}.

There has been growing interest into investigating how datacenters can exploit spatial or temporal load-shifting flexibility to achieve certain economical or environmental goals. Many studies have proposed workload scheduling algorithms for geographical load shifting to reduce datacenter electricity costs \cite{velascoElasticOperationsFederated2014, douCarbonAwareElectricityCost2017, heMinimizingOperationCost2021, raoDistributedCoordinationInternet2012, renCarbonAwareEnergyCapacity2012, dengEcoAwareOnlinePower2016}, or looked at the potential of datacenter load migration for mitigating renewable curtailment and \gls{ghg} emissions in the local electricity grid \cite{zhengMitigatingCurtailmentCarbon2020, mahmudDistributedFrameworkCarbon2016}, or aimed to increase a share of renewable energy in datacenter electricity consumption \cite{wangGreenawareVirtualMachine2015, kimDataCentersDispatchable2017, liuGeographicalLoadBalancing2011, kellyBalancingPowerSystems2016}.

Recently, several papers elaborated a mathematical problem that captures both spatial and temporal load-shifting flexibility. Zhang et al. \cite{zhangFlexibilityNetworksData2020} introduced the concept of virtual links to capture space-time load flexibility provided by geographically-distributed data centers in market clearing procedures. This work was followed by a follow-up paper, where authors elaborated an electricity market clearing formulation that seeks to remunerate spatio-temporal load shifting for the flexibility service datacenters can offer to the power grid \cite{zhangRemuneratingSpaceTime2022}.

% **Data center special features as electricity consumers*

The growing demand for computing drives economies of scale. Large companies, such as Amazon, Google, and Misrosoft are centralizing computing infrastructure into the \enquote{hyperscale} data centres. 
Such large and highly efficient facilities consist of hundreds of server nodes, that are managed collectively and geographically dispersed throughout the world \cite{ThereAre500}. 

In their role as electricity consumers, hyperscale data centres have unique characteristics when it comes to demand-side management. Data centre operators have the ability to shift \enquote{a considerable fraction} of computing jobs \textit{in~time and space}. \cite{radovanovicIEEE2023} The temporal load shifting implies re-scheduling of flexible workloads to another time point, i.e., delaying computing job execution. The spatial load shifting implies migration of flexible compute jobs and associated power loads across physical data centre locations. These characteristics equip data center operators with the unique ability to provide space-time load-shifting flexibility to the power grid.

Climate change is driving a global effort to rapidly decarbonise electricity systems across the globe.

% 2. **Problem**
text

% 5. **Literature review**

\textbf{Existing literature --}
text

% 6. **Relevance of this paper**

\textbf{Contribution --} The novelty of this study is

% 7. **Prologue**

The remainder of the paper is structured as follows: \cref{sec:methods} introduces a mathematical model of clean energy procurement strategies and gives a brief summary of the methodology, sources of model input data, and the experimental setup.
