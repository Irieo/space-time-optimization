% SPDX-FileCopyrightText: 2023 Iegor Riepin, Tom Brown
%
% SPDX-License-Identifier: CC-BY-4.0

% Introduction Structure

% 1.1 Opening: ICT sector energy consumption and emissions

Modern technology has impacted almost every aspect of everyday life, leading to a rapidly growing demand for digital services. Internet users have doubled worldwide over the past decade, while traffic has grown 25-fold. \cite{ieaDataCentresData2023}
The significant energy consumption of the Information and Communications Technology (\gls{ict}) infrastructure is becoming a concern from the ecological standpoint.
Global datacenter energy use is estimated to be between 1\% and 2\% of the final electricity demand worldwide in 2020 \cite{davidmyttonHowMuchEnergy2020, masanetRecalibratingGlobalData2020} and is likely to increase rapidly in the future \cite{andraeGlobalElectricityUsage2015}. 
The Greenhouse Gas (\gls{ghg}) emissions footprint of the datacenters and data transmission networks was estimated at 330~million tones (Mt) \co-eq in 2020, equivalent to 0.9\% of global energy-related emissions. \cite{ieaDataCentresData2023, malmodinICTSectorElectricity2023} 
For perspective, the national CO$_2$ emissions of the United Kingdom were around 331~Mt in 2021. \cite{UKnationalstats} 

% **datacenter special features as electricity consumers*

The growing demand for computing drives economies of scale. Large companies, such as Amazon, Google, and Misrosoft are centralizing computing infrastructure into the \enquote{hyperscale} datacenters. 
Such large and highly efficient facilities consist of hundreds of server nodes, that are managed collectively and geographically dispersed throughout the world \cite{ThereAre500}. 

In their role as electricity consumers, hyperscale datacenters have unique characteristics when it comes to demand-side management. 
Some computing jobs and associated power loads are \enquote{flexible}, i.e., the jobs are not time-critical and can be scheduled flexibly without impacting the overall quality of service.
Therefore, datacenter operators have the ability to shift a considerable fraction of computing workloads \textit{in~time} and \textit{in space}. \cite{radovanovicIEEE2023} 
The temporal load shifting implies re-scheduling of flexible workloads to another time point, i.e., delaying computing job execution. 
The spatial load shifting implies migration of flexible compute jobs and associated power loads across physical datacenter locations. 
These characteristics equip datacenter operators with the unique ability to provide space-time load-shifting flexibility to the power grid \cite{zhangRemuneratingSpaceTime2022} and reduce the carbon footprint of their electricity usage. \cite{radovanovicIEEE2023, douCarbonAwareElectricityCost2017}

The significant energy consumption of the Information and Communications Technology (\gls{ict}) infrastructure is becoming a concern from the ecological standpoint. Global data center energy use is estimated to be between 1\% and 2\% of the final electricity demand worldwide in 2020 \cite{davidmyttonHowMuchEnergy2020, masanetRecalibratingGlobalData2020} and is likely to increase rapidly in the future \cite{andraeGlobalElectricityUsage2015}. The Greenhouse Gas (\gls{ghg}) emissions footprint of the data centres and data transmission networks was estimated at 330~million tones (Mt) \co-eq in 2020, equivalent to 0.9\% of global energy-related emissions. \cite{ieaDataCentresData2023, malmodinICTSectorElectricity2023} For perspective, the national CO$_2$ emissions of the United Kingdom were around 331~Mt in 2021. \cite{UKnationalstats} 

% **Data center special features as electricity consumers*

The growing demand for computing drives economies of scale. Large companies, such as Amazon, Google, and Misrosoft are centralizing computing infrastructure into the \enquote{hyperscale} data centres. 
Such large and highly efficient facilities consist of hundreds of server nodes, that are managed collectively and geographically dispersed throughout the world \cite{ThereAre500}. 

In their role as electricity consumers, hyperscale data centres have unique characteristics when it comes to demand-side management. Data centre operators have the ability to shift \enquote{a considerable fraction} of computing jobs \textit{in~time and space}. \cite{radovanovicIEEE2023} The temporal load shifting implies re-scheduling of flexible workloads to another time point, i.e., delaying computing job execution. The spatial load shifting implies migration of flexible compute jobs and associated power loads across physical data centre locations. These characteristics equip data center operators with the unique ability to provide space-time load-shifting flexibility to the power grid.

Climate change is driving a global effort to rapidly decarbonise electricity systems across the globe.

% 2. **Problem**
text

% 5. **Literature review**

\textbf{Existing literature --}
text

% 6. **Relevance of this paper**

\textbf{Contribution --} The novelty of this study is

% 7. **Prologue**

The remainder of the paper is structured as follows: \cref{sec:methods} introduces a mathematical model of clean energy procurement strategies and gives a brief summary of the methodology, sources of model input data, and the experimental setup.
