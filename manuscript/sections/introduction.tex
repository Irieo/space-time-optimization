% SPDX-FileCopyrightText: 2023 Iegor Riepin, Tom Brown
%
% SPDX-License-Identifier: CC-BY-4.0

% Introduction Structure

% 1.1 Opening: ICT sector energy consumption and emissions

Modern technology has impacted almost every aspect of everyday life, leading to a rapidly growing demand for digital services. Internet users have doubled worldwide over the past decade, while traffic has grown 25-fold. \cite{ieaDataCentresData2023}

The significant energy consumption of the Information and Communications Technology (\gls{ict}) infrastructure is becoming a concern from the ecological standpoint. Global data center energy use is estimated to be between 1\% and 2\% of the final electricity demand worldwide in 2020 \cite{davidmyttonHowMuchEnergy2020, masanetRecalibratingGlobalData2020} and is likely to increase rapidly in the future \cite{andraeGlobalElectricityUsage2015}. The Greenhouse Gas (\gls{ghg}) emissions footprint of the data centres and data transmission networks was estimated at 330~million tones (Mt) \co-eq in 2020, equivalent to 0.9\% of global energy-related emissions. \cite{ieaDataCentresData2023, malmodinICTSectorElectricity2023} For perspective, the national CO$_2$ emissions of the United Kingdom were around 331~Mt in 2021. \cite{UKnationalstats} 

% **Data center special features as electricity consumers*

The growing demand for computing drives economies of scale. Large companies, such as Amazon, Google, and Misrosoft are centralizing computing infrastructure into the \enquote{hyperscale} data centres. 
Such large and highly efficient facilities consist of hundreds of server nodes, that are managed collectively and geographically dispersed throughout the world \cite{ThereAre500}. 

In their role as electricity consumers, hyperscale data centres have unique characteristics when it comes to demand-side management. Data centre operators have the ability to shift \enquote{a considerable fraction} of computing jobs and associated power loads \textit{in~time} via scheduling of flexible compute jobs and \textit{in~space} via migration of flexible compute jobs across physical locations. \cite{radovanovicCarbonAwareComputingDatacenters2023} These characteristics equip data center operators with the unique ability to provide space-time load-shifting flexibility to the power grid.

Climate change is driving a global effort to rapidly decarbonise electricity systems across the globe.

% 2. **Problem**
text

% 5. **Literature review**

\textbf{Existing literature --}
text

% 6. **Relevance of this paper**

\textbf{Contribution --} The novelty of this study is

% 7. **Prologue**

The remainder of the paper is structured as follows: \cref{sec:methods} introduces a mathematical model of clean energy procurement strategies and gives a brief summary of the methodology, sources of model input data, and the experimental setup.
