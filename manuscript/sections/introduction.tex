% SPDX-FileCopyrightText: 2023 Iegor Riepin, Tom Brown
%
% SPDX-License-Identifier: CC-BY-4.0

% Introduction Structure

% 1.1 Opening: ICT sector energy consumption and emissions

Modern technology has impacted almost every aspect of everyday life, leading to a rapidly growing demand for digital services. Internet users have doubled worldwide over the past decade, while traffic has grown 25-fold. \cite{ieaDataCentresData2023}
The significant energy consumption of the Information and Communications Technology (\gls{ict}) infrastructure is becoming a concern from the ecological standpoint.
Global datacenter energy use is estimated to be between 1\% and 2\% of the final electricity demand worldwide in 2020 \cite{davidmyttonHowMuchEnergy2020, masanetRecalibratingGlobalData2020} and is likely to increase rapidly in the future \cite{andraeGlobalElectricityUsage2015}. 
The Greenhouse Gas (\gls{ghg}) emissions footprint of the datacenters and data transmission networks was estimated at 330~million tones (Mt) \co-eq in 2020, equivalent to 0.9\% of global energy-related emissions. \cite{ieaDataCentresData2023, malmodinICTSectorElectricity2023} 
For perspective, the national CO$_2$ emissions of the United Kingdom were around 331~Mt in 2021. \cite{UKnationalstats} 

% **datacenter special features as electricity consumers*

The growing demand for computing drives innovation and economies of scale. Large companies, such as Amazon, Google, and Misrosoft are centralizing computing infrastructure into the \enquote{hyperscale} datacenters. 
Such large and highly efficient facilities consist of hundreds of server nodes, that are managed collectively and geographically dispersed throughout the world \cite{ThereAre500}. 

In their role as electricity consumers, hyperscale datacenters have unique characteristics when it comes to demand-side management. 
Some computing jobs and associated power loads are \enquote{flexible}, i.e., the jobs are not time-critical and can be scheduled flexibly without impacting the overall quality of service.
Therefore, datacenter operators have the ability to shift a considerable fraction of computing workloads \textit{in~time} and \textit{in space}. \cite{radovanovicIEEE2023} 
The temporal load shifting implies re-scheduling of flexible workloads to another time point, i.e., delaying computing job execution. 
The spatial load shifting implies migration of flexible compute jobs and associated power loads across physical datacenter locations. 

The significant energy consumption of \gls{ict} infrastructure, and its anticipated growth, raise a natural question: \textit{How can datacenters make use of unique spatial and temporal load-shifting capabilities to improve their economics and the environmental footprint?} The topic of this overarching question has been the subject of a growing body of literature over the past decade.

% Literature review
\textbf{Existing literature --} The work by Wierman et al.  \cite{wiermanOpportunitiesChallengesData2014} surveyed the forms, opportunities and challenges for data center demand response options, including aspects of temporal load shifting such as workload delaying or shedding and the geographical load balancing. Several papers discussed methods and policies needed to incentivise parcicipation of datacenters in the power grid's demand responce \cite{liuPricingDataCenter2014, zhouTruthfulEfficientIncentive2020}.

Numerous studies investigated ways to harness spatial or temporal load-shifting flexibility to achieve certain economical or environmental goals. Many studies have proposed workload scheduling algorithms for geographical load shifting to reduce datacenter electricity costs \cite{velascoElasticOperationsFederated2014, douCarbonAwareElectricityCost2017, heMinimizingOperationCost2021, raoDistributedCoordinationInternet2012, renCarbonAwareEnergyCapacity2012, dengEcoAwareOnlinePower2016}, or looked at the potential of datacenter load migration for mitigating renewable curtailment and \gls{ghg} emissions in the local electricity grid \cite{zhengMitigatingCurtailmentCarbon2020, mahmudDistributedFrameworkCarbon2016}, or aimed to increase a share of renewable energy in datacenter electricity consumption \cite{wangGreenawareVirtualMachine2015, kimDataCentersDispatchable2017, liuGeographicalLoadBalancing2011, kellyBalancingPowerSystems2016}.

Recently, several papers elaborated a mathematical problem that captures both spatial and temporal load-shifting flexibility. Zhang et al. \cite{zhangFlexibilityNetworksData2020} introduced the concept of virtual links to capture space-time load flexibility provided by geographically-distributed data centers in market clearing procedures. This work was followed by a follow-up paper, where authors elaborated an electricity market clearing formulation that seeks to remunerate spatio-temporal load shifting for the flexibility service datacenters can offer to the power grid \cite{zhangRemuneratingSpaceTime2022}.

A great deal of attention in previous research exploring the drivers for geograhical load shifting has focused on signals provided by the electricity grids where datacenters operate. These signals include average carbon emissions in the region of operation \cite{zhengMitigatingCurtailmentCarbon2020}, or locational marginal carbon emissions \cite{lindbergEnvironmentalPotentialHyperScale2021}, or locational electricity prices and price differences across datacenter locations \cite{raoMinimizingElectricityCost2010,tranHowGeoDistributedData2016, zhangRemuneratingSpaceTime2022}. This emerging stream of research work is now finding practical applications. For example, Google recently introduced a Carbon-Intelligent Compute Management system that minimizes carbon footprint and power infrastructure costs by shifting flexible workloads in time  across datacenter fleet as a function to the next day's carbon intensity forecasts. \cite{radovanovicIEEE2023} There are several companies providing market data and forecasts on carbon emissions and electricity prices, such as WattTime \cite{WattTime}, ElectricityMaps \cite{ElectricityMaps}, and others.

% 24/7 CFE**
\textbf{24/7 Carbon-Free Energy--} The \gls{ict} companies invest significantly in renewable energy to reduce their environmental impact, reduce power price volatility, and enhance their brand image. The hyperscale datacenter industry in particular leads in corporate renewable energy acquisition through Power Purchase Agreements (\gls{ppa}s). With almost 50~GW contracted so far, Amazon, Microsoft, Meta, and Google are the four largest purchasers of corporate renewable energy \gls{ppa}s. \cite{ieaDataCentresData2023} Furthermore, the leaders in corporate clean energy procurement pledge to eliminate \textit{all} greenhouse gas emissions associated with their electricity use. These companies declare the \enquote{24/7 Carbon-Free Energy (\gls{cfe})} goal aiming to match electricity demand with clean energy supply on an \textit{hourly basis}. The 24/7 targets were announced by Google, Microsoft, and Iron Mountain for 2030 and 2040, respectively \cite{google-247by2030, Microsoft-vision, IronMountainSustainability}.

One notable feature of the 24/7 CFE goal is that it drives---and requires---a significant volume of energy procurement via \gls{ppa}s. Recent research showed that achieving 24/7 CFE with a 100\% target (i.e., achiving 0~gCO$_2$ emissions per kWh) implies that companies need to rely on own procured resources, void of any electricity producerement from the local grid. \cite{riepinMeansCostsSystemlevel2023, riepin-zenodo-systemlevel247} It is because the local electricity mix in most regions and at most times does not have a strictly zero carbon content and therefore cannot contribute to 24/7 CFE.

% The Problem**
\textbf{The problem--} Hyperscale datacenter operators trying to reduce their carbon footprint with clean energy procurement are facing a situation that grid signals such as carbon emissions and locational electricity prices lose relevance for informed load shifting. This effect takes place when a company requires less electricity from the local grid when relying more on its own portfolio of carbon-free generators and storage. This effect was illustrated with the help of energy system modelling in the previous work of the authours although was not investigated or discussed in detail. \cite{riepinValueSpacetimeLoadshifting2023} In this context, the following question arise: \textit{In pursuit of achieving 24/7 Carbon-Free Energy objectives, what key signals should datacenter operators focus on to facilitate informed and effective geographical load shifting? }

% 6. **Relevance of this paper**
% TODO: adjust if temporal shifting sneaks stody sneaks into the scope of the paper.
\textbf{Contribution --} The aim of this paper is to identify signals for effective geographical load shifting in the context of the 24/7 CFE matching. 

For that, we develop a computer model that simulates a datacenter fleet with controlled degrees of flexibility in pursuit of 24/7 CFE objectives. Datacenter fleet optimization is incorporated into the European electricity system model as a mathematical problem.
The information is identified and split into individual signals that are relevant for effective geographical load shifting. Signals include (a) lags in solar radiation due to Earth rotation, (b) low correlation between wind power generation over long distances due to different weather conditions, and (c) varying quality of renewable energy resources. It is evident that each of these signals is important for effective geographical load shifting. Furthermore, we demonstrate that load shifting decisions optimized considering all of these signals facilitate 24/7 CFE matching efficiency and affordability. When all the signals are taken into account, geographical load-shifting can reduce the electricity costs of a datacenter fleet by X\% for each percent point of flexible load.

All code necessary to reproduce the experiments is published under an open license alongside the paper. See \nameref{sec:code}.

% 7. **Prologue**

The remainder of the paper is structured as follows: \cref{sec:methods} introduces a mathematical model of clean energy procurement strategies and gives a brief summary of the methodology, sources of model input data, and the experimental setup.
